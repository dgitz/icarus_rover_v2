\href{https://travis-ci.org/mavlink/mavlink}{\tt }

\subsection*{M\+A\+V\+Link}


\begin{DoxyItemize}
\item Official Website\+: \href{http://mavlink.org}{\tt http\+://mavlink.\+org}
\item Source\+: \href{https://github.com/mavlink/mavlink}{\tt Mavlink Generator}
\item Binaries (always up-\/to-\/date from master)\+:
\begin{DoxyItemize}
\item \href{https://github.com/mavlink/c_library_v1}{\tt C/\+C++ header-\/only library v1}
\item \href{https://github.com/mavlink/c_library_v2}{\tt C/\+C++ header-\/only library v2}
\end{DoxyItemize}
\item Mailing list\+: \href{http://groups.google.com/group/mavlink}{\tt Google Groups}
\end{DoxyItemize}

M\+A\+V\+Link -- Micro Air Vehicle \hyperlink{structMessage}{Message} Marshalling Library.

This is a library for lightweight communication between drones and/or ground control stations. It allows for defining messages within X\+ML files, which then are converted into appropriate source code for different languages. These X\+ML files are called dialects, most of which build on the {\itshape common} dialect provided in {\ttfamily common.\+xml}.

The initial experimental M\+A\+V\+Link was created 2008 when the term drone was not used yet to describe small vehicles for consumer use.

The M\+A\+V\+Link protocol performs byte-\/level serialization and so is appropriate for use with any type of radio modem.

This repository is largely Python scripts that convert X\+ML files into language-\/specific libraries. There are additional Python scripts providing examples and utilities for working with M\+A\+V\+Link data. These scripts, as well as the generated Python code for M\+A\+V\+Link dialects, require Python 2.\+7 or greater.

\subsubsection*{Requirements}


\begin{DoxyItemize}
\item Python 2.\+7+
\begin{DoxyItemize}
\item Tkinter (if G\+UI functionality is desired)
\end{DoxyItemize}
\end{DoxyItemize}

\subsubsection*{Installation}


\begin{DoxyEnumerate}
\item Clone into a user-\/writable directory. Make sure to use the git \char`\"{}-\/-\/recursive\char`\"{} option since pymavlink is a submodule. Alternately, run \char`\"{}git submodule init\char`\"{} and \char`\"{}git submodule update\char`\"{} after cloning to get pymavlink.
\item Add the repository directory to your {\ttfamily P\+Y\+T\+H\+O\+N\+P\+A\+TH}
\item Generate M\+A\+V\+Link parser files following the instructions in the next section {\itshape A\+N\+D/\+OR} run included helper scripts described in the Scripts/\+Examples secion.
\end{DoxyEnumerate}

\subsubsection*{Generating Language-\/specific Source Files}

Language-\/specific files can be generated via a Python script from the command line or using a G\+UI. If a dialect X\+ML file has a dependency on another X\+ML file, they must be located in the same directory. Since most M\+A\+V\+Link dialects depend on the {\bfseries common} messageset, it is recommend that you place your dialect with the others in {\ttfamily /message\+\_\+definitions/v1.0/}.

Available languages are\+:


\begin{DoxyItemize}
\item C
\item C\#
\item Java
\item Java\+Script
\item Lua
\item Python, version 2.\+7+
\end{DoxyItemize}

\paragraph*{From a G\+UI (recommended)}

mavgenerate.\+py is a header generation tool G\+UI written in Python. It requires Tkinter, which is only included with Python on Windows, so it will need to be installed separately on non-\/\+Windows platforms. It can be run from anywhere using Python\textquotesingle{}s -\/m argument\+: \begin{DoxyVerb}$ python -m mavgenerate
\end{DoxyVerb}


\paragraph*{From the command line}

mavgen.\+py is a command-\/line interface for generating a language-\/specific M\+A\+V\+Link library. This is actually the backend used by {\ttfamily mavgenerate.\+py}. After the {\ttfamily mavlink} directory has been added to the Python path, it can be run by executing from the command line\+: \begin{DoxyVerb}$ python -m pymavlink.tools.mavgen
\end{DoxyVerb}


\subsubsection*{Usage}

Using the generated M\+A\+V\+Link dialect libraries varies depending on the language, with language-\/specific details below\+:

\paragraph*{C}

To use M\+A\+V\+Link, include the {\itshape mavlink.\+h} header file in your project\+: \begin{DoxyVerb}#include <mavlink.h>
\end{DoxyVerb}


Do not include the individual message files. In some cases you will have to add the main folder to the include search path as well. To be safe, we recommend these flags\+: \begin{DoxyVerb}$ gcc -I mavlink/include -I mavlink/include/<your message set, e.g. common>
\end{DoxyVerb}


The C M\+A\+V\+Link library utilizes a channels metaphor to allow for simultaneous processing of multiple M\+A\+V\+Link streams in the same program. It is therefore important to use the correct channel for each operation as all receiving and transmitting functions provided by M\+A\+V\+Link require a channel. If only one M\+A\+V\+Link stream exists, channel 0 should be used by using the {\ttfamily M\+A\+V\+L\+I\+N\+K\+\_\+\+C\+O\+M\+M\+\_\+0} constant.

\subparagraph*{Receiving}

M\+A\+V\+Link reception is then done using {\ttfamily mavlink\+\_\+helpers.\+h\+:mavlink\+\_\+parse\+\_\+char()}.

\subparagraph*{Transmitting}

Transmitting can be done by using the {\ttfamily mavlink\+\_\+msg\+\_\+$\ast$\+\_\+pack()} function, where one is defined for every message. The packed message can then be serialized with {\ttfamily mavlink\+\_\+helpers.\+h\+:mavlink\+\_\+msg\+\_\+to\+\_\+send\+\_\+buffer()} and then writing the resultant byte array out over the appropriate serial interface.

It is possible to simplify the above by writing wrappers around the transmitting/receiving code. A multi-\/byte writing macro can be defined, {\ttfamily M\+A\+V\+L\+I\+N\+K\+\_\+\+S\+E\+N\+D\+\_\+\+U\+A\+R\+T\+\_\+\+B\+Y\+T\+E\+S()}, or a single-\/byte function can be defined, {\ttfamily comm\+\_\+send\+\_\+ch()}, that wrap the low-\/level driver for transmitting the data. If this is done, {\ttfamily M\+A\+V\+L\+I\+N\+K\+\_\+\+U\+S\+E\+\_\+\+C\+O\+N\+V\+E\+N\+I\+E\+N\+C\+E\+\_\+\+F\+U\+N\+C\+T\+I\+O\+NS} must be defined.

\subsubsection*{Scripts/\+Examples}

This M\+A\+V\+Link library also comes with supporting libraries and scripts for using, manipulating, and parsing M\+A\+V\+Link streams within the pymavlink, pymav link/tools, and pymavlink/examples directories.

The scripts have the following requirements\+:
\begin{DoxyItemize}
\item Python 2.\+7+
\item mavlink repository folder in {\ttfamily P\+Y\+T\+H\+O\+N\+P\+A\+TH}
\item Write access to the entire {\ttfamily mavlink} folder.
\item Your dialect\textquotesingle{}s X\+ML file is in {\ttfamily message\+\_\+definitions/$\ast$/}
\end{DoxyItemize}

Running these scripts can be done by running Python with the \textquotesingle{}-\/m\textquotesingle{} switch, which indicates that the given script exists on the P\+Y\+T\+H\+O\+N\+P\+A\+TH. This is the proper way to run Python scripts that are part of a library as per P\+E\+P-\/328 (and the rejected P\+E\+P-\/3122). The following code runs {\ttfamily mavlogdump.\+py} in {\ttfamily /pymavlink/tools/} on the recorded M\+A\+V\+Link stream {\ttfamily test\+\_\+run.\+mavlink} (other scripts in {\ttfamily /tools} and {\ttfamily /scripts} can be run in a similar fashion)\+: \begin{DoxyVerb}$ python -m pymavlink.tools.mavlogdump test_run.mavlink
\end{DoxyVerb}


\subsubsection*{License}

M\+A\+V\+Link is licensed under the terms of the Lesser General Public License (version 3) of the Free Software Foundation (L\+G\+P\+Lv3). The C-\/language version of M\+A\+V\+Link is a header-\/only library which is generated as M\+I\+T-\/licensed code. M\+A\+V\+Link can therefore be used without limits in any closed-\/source application without publishing the source code of the closed-\/source application.

See the {\itshape C\+O\+P\+Y\+I\+NG} file for more info.

\subsubsection*{Credits}

\copyright{} 2009-\/2014 \href{mailto:mail@qgroundcontrol.org}{\tt Lorenz Meier} 